\documentclass[11pt,a4paper]{article}
% Note draft will skip inclusion/graphics but is needed for annotations
\usepackage[draft]{pdfcomment}
\usepackage{lipsum}
\usepackage{url}
\usepackage{stix2}
\title{Test Document for PDF Annotations and Grading}
\author{Joseph Timothy Foley}
\date{\today}
% TODO: Convert to teaching tools documentation?
% TODO: Urls and hyperref
\usepackage{hyperref}
\begin{document}
\maketitle{}
\newcommand{\texmacro}[1]{\texttt{\textbackslash#1}}
\section{Introduction}
This is the introduction.
We will be using the \path{pdfcomment.sty} \LaTeX{} package to automatically generate some annotations to start.
Documentation for the package at \url{https://ctan.uib.no/macros/latex/contrib/pdfcomment/doc/pdfcomment.pdf}
On this sentence, we use \texmacro{pdfsidelinecomment}\pdfsidelinecomment{Here's the sideline comment!} for instance which I am not sure the annotation detector will find.

For putting comments at some point in the text, we use \texmacro{pdfcomment}\pdfcomment{A PDF comment, yay!}.
Perhaps you prefer them to go in the margin; then you used use \texmacro{pdfmargincomment}\pdfmargincomment{Marginal comments work like this.}

\texmacro{pdfmarkupcomment} is a more general macro:
To highlight something of interest \pdfmarkupcomment[markup=Highlight]{like a particular memory of a rose petal,}{TODO:\@ figure out how to change the highlight color to a nice yellow} use the markup option Highlight.
To underline for \pdfmarkupcomment[markup=Squiggly]{emphasis from the editor}{The squiggly will make you pay attention!} use the markup option Squiggly.
Perhaps a word is spelled \pdfmarkupcomment[markup=StrikeOut]{incorecctly}{Spelled it wrong!}, then you would probably give it the markup option StrikeOut.
If you just want to put a comment somewhere random on the page\pdffreetextcomment[type=freetext, voffset=-1.5cm, hoffset=-1cm, height=1cm, width=5cm]{Not quite random, it does need to go somewhere.}, you can use \texmacro{pdffreetextcomment} with the option type freetext. 

\section{Testing}
Now let's get to some real test comments for the attached grading sheet.\pdfcomment{FOR1: Basic deduction}
\pdfcomment{FOR1 Deductions}
\pdfcomment{FOR1 Deductions}
\pdfcomment{FOR1 Deductions}
\pdfcomment{FOR1(-2) Deductions (negative)}
\pdfcomment{FOR2(0.5) Halfsies}
\pdfcomment{FOR2(0.5) Halfsies}
\pdfcomment{FOR2(0.5) Halfsies}
\pdfcomment{FOR3(2) Deduction (still negative)}
\pdfcomment{FOR4(=10) Set value}
\pdfcomment{PRO1(*) Deduct all points}

\subsection{Penalties}
\pdfcomment{CITE! Basic penalty}
\pdfcomment{LATE!(2) Double penalty}
\pdfcomment{CITE Penalty, but forgot the exclamation point}
\subsection{Wierdness}
\pdfcomment{PRO2 Out of order deduction}
\pdfcomment{PRO Maybe wrongly formatted deduction?}
\pdfcomment{SP Spelled something else wrong}
\subsection{Colons matter, maybe?}
\pdfcomment{PRO3: Colon deduction}
\pdfcomment{PRO4(2): Colon parenthetical deduction}
\subsection{Advanced stuff}
\pdfcomment{ALL*(*) Ignore all of the grades in ALL}
\pdfcomment{PRO5(SKIP):  Special command, skip grading}

How about multiline comments\pdfcomment{RND1(SKIP): Can we do multiline?\textCR\textLF RND2(SKIP): This is the second line}

%% TODO:  make an all section or something similar

\section{Design}
\lipsum[3]
\section{Prototyping}
\lipsum[4]
\section{Conclusion}
\lipsum[5]
\end{document}

%%% Local Variables:
%%% mode: latex
%%% TeX-master: t
%%% End:
